\chapter*{Lời cảm ơn}
\addcontentsline{toc}{chapter}{Lời cảm ơn}

Trong quá trình thực hiện đề tài 
"Nhận diện từ loại trong văn bản Tiếng Việt" 
thuộc môn Học máy, nhóm chúng em đã nhận được sự 
hỗ trợ và giúp đỡ quý báu từ nhiều phía. Những 
kiến thức lý thuyết trên lớp, cùng với sự hướng 
dẫn tận tình và những giờ học thực hành, đã 
trở thành nguồn động lực to lớn giúp chúng em 
hoàn thành đề tài này. Đầu tiên, chúng em xin 
gửi lời cảm ơn chân thành đến 
TS. Cao Văn Chung, ThS. Hà Mỹ Linh, 
Cử nhân Lê Ngọc Toàn và Cử nhân Phạm Thị Đức, 
giảng viên Trường Đại học Khoa học Tự Nhiên, 
những người đã trực tiếp hướng dẫn và luôn sẵn 
sàng hỗ trợ nhóm chúng em. Thầy cô đã không ngần 
ngại dành thời gian và công sức để hướng dẫn 
chi tiết từng bước, từ việc lựa chọn phương pháp 
nghiên cứu cho đến việc phân tích kết quả. 
Những buổi gặp gỡ và trao đổi với thầy cô không 
chỉ giúp chúng em hiểu rõ hơn về các khái niệm 
lý thuyết mà còn truyền cảm hứng cho chúng em 
trong quá trình thực hiện đề tài. Chúng em cũng 
xin gửi lời cảm ơn đến các thầy cô giáo khác 
trong khoa, những người đã truyền đạt cho chúng 
em những kiến thức quý giá và tạo ra một môi 
trường học tập tích cực. Sự nhiệt huyết và tâm 
huyết của các thầy cô đã góp phần không nhỏ 
vào sự hình thành và phát triển của chúng em. 
Cuối cùng, chúng em muốn cảm ơn những người bạn 
cùng lớp, những người đã cùng nhau chia sẻ tài 
liệu, kiến thức và cùng nhau học tập. Chúng em 
xin chân thành cảm ơn tất cả mọi người đã đồng 
hành cùng chúng em trong hành trình này. Sự giúp 
đỡ và ủng hộ của mọi người chính là động lực lớn 
lao để chúng em tiếp tục phấn đấu và phát triển 
trong tương lai.
% \vspace{1cm}
% \begin{flushright}
% \textit{Hà Nội, tháng 01 năm 2025}\\[0.5cm]
% \textbf{Nhóm sinh viên thực hiện}\\
% % Cao Hải An\\
% % Đặng Thế Anh\\
% % Đỗ Minh Đức
% \end{flushright}
